%This is just a generated template
\section{Methods}

To classify gut microbiome health states, we employ a machine learning pipeline that integrates metagenomic and fluxomic data. Our methodology consists of the following steps:

\subsection{Data Acquisition}
Metagenomic sequencing data is collected from publicly available microbiome datasets, while fluxomic simulations are generated using genome-scale metabolic models (GEMs). The datasets are preprocessed to remove contaminants and ensure quality control.

\subsection{Preprocessing}
Metagenomic reads undergo taxonomic and functional profiling using tools like Kraken2 and HUMAnN. Fluxomic data is generated using constraint-based modeling approaches, such as flux balance analysis (FBA), to predict metabolic flux distributions.

\subsection{Feature Engineering and Selection}
Key microbial and metabolic features are extracted using dimensionality reduction techniques like principal component analysis (PCA) and feature selection algorithms such as recursive feature elimination (RFE). These features serve as inputs for the ML models.

\subsection{Model Training and Evaluation}
Supervised machine learning models, including Random Forest, XGBoost, and neural networks, are trained to classify healthy and dysbiotic microbiome states. Model performance is assessed using cross-validation and evaluation metrics such as accuracy, precision, recall, F1-score, and the area under the receiver operating characteristic curve (AUC-ROC).
